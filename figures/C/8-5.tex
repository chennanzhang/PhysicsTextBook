\documentclass{standalone}
\usepackage{tikz}
\usepackage{ctex,siunitx}
\setCJKmainfont{Noto Serif CJK SC}
\usepackage{tkz-euclide}
\usepackage{amsmath}
\usetikzlibrary{patterns, calc}
\usetikzlibrary {decorations.pathmorphing, decorations.pathreplacing, decorations.shapes,}
\begin{document}
\small
\begin{tikzpicture}[>=latex,scale=1.6]
  \foreach \x[count=\i] in {5,15,30,50,75,100}
  {
    \draw(0,0)circle(\x pt);
    \node at (0,-\x pt)[below]{$n=\i$};
  }
  \node at (60:52pt) [rotate=60,above] {赖曼系(紫外线)};
  \node at (10:50pt)[rotate=10,above] {巴耳末系};
  \node at (-20:77pt)[rotate=-20,above] {帕邢系(红外线)};
  \node at (-45:87pt)[rotate=-45,above] {布喇开系};
  \node at (-60:90pt)[left] {逢德系};
  \draw[->] (60:100pt)--(60:5pt);
  \draw[->] (55:75pt)--(55:5pt);
  \draw[->] (50:50pt)--(50:5pt);
  \draw[->] (45:30pt)--(45:5pt);
  \draw[->] (40:15pt)--(40:5pt);
  \draw[->] (10:100pt)--(10:15pt);
  \draw[->] (5:75pt)--(5:15pt);
  \draw[->] (0:50pt)--(0:15pt);
  \draw[->] (-5:30pt)--(-5:15pt);
  \draw[->] (-20:100pt)--(-20:30pt);
  \draw[->] (-25:75pt)--(-25:30pt);
  \draw[->] (-30:50pt)--(-30:30pt);
  \draw[->] (-45:100pt)--(-45:50pt);
  \draw[->] (-50:75pt)--(-50:50pt);
  \draw[->] (-60:100pt)--(-60:75pt);
\end{tikzpicture}
\end{document}
\documentclass{standalone}
\usepackage{tikz}
\usepackage{ctex,siunitx}
\setCJKmainfont{Noto Serif CJK SC}
\usepackage{tkz-euclide}
\usepackage{amsmath}
\usetikzlibrary{patterns, calc}
\usetikzlibrary {decorations.pathmorphing, decorations.pathreplacing, decorations.shapes,}
\begin{document}
\small
\begin{tikzpicture}[>=latex,scale=1.0]
  % \useasboundingbox(-2,-2.2)rectangle(2,1.5);
  \draw[pattern=north east lines](-0.05,-1.5)rectangle(0.05,-0.5)node[below left] {$-$}(-0.05,0.67)node[above left]{$+$}rectangle(0.05,1.5);
  \draw[pattern=crosshatch dots](-1.5,-0.5)--(-1.5,0.5)--(-0.45,0.5)to[bend right=30](0.45,0.5)--(0.6,0.5)--(1.5,0.5)--(1.5,-0.5)--cycle;
  \fill(-0.6,0.5)--(-0.45,0.5)to[bend right=30](0.45,0.5)--(0.6,0.5)to[bend left=50](-0.6,0.5);
  \fill[pattern =north east lines,draw](-0.4,0.5)to[bend left=50](0.4,0.5)to[bend left=25](-0.4,0.5);
  \draw[thin](-0.05,1.0)--++(150:0.3)node[left]{引线};
  \draw[thin](-0.05,-1.0)--++(210:0.3)node[left]{引线};
  \draw[thin](0.55,-0.3)--++(-60:0.5)node[below right]{N 型半导体};
  \draw[thin](0.55,0.5)--++(60:0.5)node[above right]{P 型半导体};
  \end{tikzpicture}
\end{document}

\chapter*{课外实验活动}
\addcontentsline{toc}{chapter}{课外实验活动}
\setcounter{section}{0}
\section{自制指南针}

\medskip\noindent
\begin{minipage}{0.75\linewidth}\parindent2em
如\cref{fig:10-10} 所示,用硬纸板、大头针、按扣、缝衣针自制一个指南针。

用磁铁的一端在缝衣针上朝一个方向擦几下,缝衣针就有了磁性。
为了使缝衣针能顺利地穿过按扣(取按扣中较薄的一扇)的两个小孔,可用钳子把按扣的边缘向下夹一下。
当自制的指南针静下来后,记住针的哪一端指北。
\end{minipage}\hfill
\begin{minipage}{0.2\linewidth}\centering
\begin{figurehere}
  \includegraphics{10-10.pdf}
  \caption{}\label{fig:10-10}
\end{figurehere}
\end{minipage}


\section{验证环形电流的磁场}
这个实验是用自制的指南针来验证环形电流的磁场方向(\cref{fig:10-11})。

\medskip\noindent
\begin{minipage}{0.6\linewidth}\parindent2em
在一个瓶子(或硬纸筒)上用漆包线绕一个 10 至 15 匝的线圈,把绕好的线圈从瓶子上取下来,再用胶布把线圆竖直固定在一块六板上。将你自制的指南针放在\cref{fig:10-11} 所示的位置,转动木板使磁针处在线圈平面内。用学过的环形电流磁场的知识判断一下,如果线圈的两端接上电池,指南针将怎样偏转,然后再给线圈通电,看一看实验结果跟你的判断是否一致。
\end{minipage}\hfill
\begin{minipage}{0.38\linewidth}\centering
  \begin{figurehere}
    \includegraphics{10-11.pdf}
    \caption{}\label{fig:10-11}
  \end{figurehere}
\end{minipage}

\section{验证通电螺线管的南北极}
把漆包线绕在一支铅笔上,然后抽出铅笔,做成一个螺线管。
用学过的通电螺线管磁场的知识判断一下,如果给螺线管通电,通电螺线管哪端是南极,哪端是北极。
然后把自制的指南针放在螺线管的两端,给螺线管通电,看看实验结果跟你的判断是否一致。

\section{观察磁化现象}
取一个条形磁铁和一个大铁钉,把铁钉插入铁屑,并把条形磁铁的一个磁极靠近钉子头。然后同时提起磁铁和铁钉,你将看到一些铁屑粘到钉子上。
将磁铁移去,铁钉上的大部分铁屑将掉下来,但仍有一部分铁屑粘在钉子上。
再用磁铁的另一个磁极靠近钉子头,剩在钉子上的铁屑就会掉下来。

解释上述现象。

\section{判断指南针的偏转方向}
在一个铅笔刀或一个大些的铁钉上,用漆包线绕上两个线圈 $A$ 和 $B$,将线圈 $B$ 的两端接在一起,并把 $CD$ 段漆包线放在静止的自制指南针的上方(\cref{fig:10-12})。
试判断当用于电池给线圈 $A$ 通电的一瞬间,指南针偏转的方向。
做这个实验,看一看你判断的指南针偏转方向与实验是否一致。
\begin{figure}
  \includegraphics{10-12.pdf}
  \caption{}\label{fig:10-12}
\end{figure}

\section{自制测电笔}
准备一个小氖灯,一个小弹簧。
再找一个装中药片的小玻璃瓶,两个瓶盖,两个铁钉,一个\qty{0.25}{W}、\qtyrange{2}{5}{M\ohm} 的电阻。
在稍粗糙的水泥砖上把玻璃瓶底磨掉,做成一个玻璃圆筒。
让铁钉穿过瓶盖,盖上瓶盖后使钉帽在瓶里。
把电阻的两根引线齐根去掉,并把电阻两端的绝缘漆去掉。
照\cref{fig:10-13} 那样把上述器材安装起来,就做成了一个测电笔。
\begin{figure}
  \includegraphics{10-13.pdf}
  \caption{自制测电笔}\label{fig:10-13}
\end{figure}

用这个自制的测电笔可以辨别照明电路的火线和地线。
用拇指和食指拿住玻璃瓶,前面的钉子接触待辨别的导线,后面的钉子接触手。
当前面的钉子接触的是火线时,小氖灯发光;接触的是地线时,小氖灯不发光。
这样就可以辨别出火线和地线。

要注意:\emph{手的任何部位都不要接触前面的钉子},因为它接触可能是火线,会使人触电。

\section{测定水的折射率}
\medskip\noindent
\begin{minipage}{0.55\linewidth}\parindent2em
找一个广口瓶,在瓶内盛满水,照\cref{fig:10-14} 那样把直尺 $AB$ 紧挨着瓶口的 $C$ 点竖直插入瓶内。
从尺的对面一点 $P$ 观察水面,可以同时看到直尺在水中的部分和露出水面的部分在水中的像。
读出你看到的直尺水下部分最低点的刻度 $S_1$,以及跟这个刻度相重合的、水上部分刻度 $S_2$ 的像 $S'_2$。
记下 $CS_1$ 和 $CS_2$ 的长度,量出广口瓶瓶口的内径 $d$,就能算出水的折射率。
你用这种方法求出的水的折射率为多少?
\end{minipage}\hfill
\begin{minipage}{0.4\linewidth}\centering
  \begin{figurehere}
    \includegraphics{10-14.pdf}
    \caption{}\label{fig:10-14}
  \end{figurehere}
\end{minipage}

\medskip
如果你能同时读出直尺在水下的两个刻度 $S_1$ 和 $S_3$,以及跟它们相重合的、两个水上刻度 $S_2$ 和 $S_4$ 在水中的像 $S'_2$ 和 $S'_4$,就可以不必测量瓶口的内径,直接用从直尺上读出的两组数据求出水的折射率来。
比较这两种方法测量的结果,看哪种方法测得的折射率更准确?

用后一种方法进行测量,瓶中的水不一定非盛满不可,竖直插入水中的直尺也不一定要紧挨瓶口,做起来更简便。

\section{测定凹透镜的焦距}
凹透镜所成的虚像不能在像屏上显示出来,因此它的焦距不可能象凸透镜那样直接利用焦点或成像方法来测量,下面介绍一种测量凹透镜焦距的简便方法。

\medskip\noindent
\begin{minipage}{0.55\linewidth}\parindent2em
在凹透镜的中心贴一个半径为 $R$ 的黑色圆纸片 $A$,另取一张白纸 $B$,在 $B$ 上画一个半径为 $2R$ 的圆。
把白纸和凹透镜平行地放在太阳光下(\cref{fig:10-15}),让透镜对着太阳,调节透镜和白纸间的距离,使黑色圆纸片的影恰好跟白纸上的圆圈重合。
这时透镜和白纸间的距离就等于凹透镜的焦距。
想想看,为什么?做这个实验,并将测得的焦距跟已知的焦距相比较,看相差多少。
\end{minipage}\hfill
\begin{minipage}{0.4\linewidth}\centering
  \begin{figurehere}
    \includegraphics{10-15.pdf}
    \caption{}\label{fig:10-15}
  \end{figurehere}
\end{minipage}
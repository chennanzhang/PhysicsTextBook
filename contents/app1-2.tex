\chapter*{课外实验活动}\markboth{课外实验活动}{课外实验活动}
\addcontentsline{toc}{chapter}{课外实验活动}
\setcounter{section}{0}
\section{测量尼龙丝的抗断拉力}
在商店里买来了尼龙丝,但不知道它的抗断拉力。
如果你手边只有一个质量为 \qty{1}{kg} 的重物和一只量角器,你能测出尼龙丝的抗断拉力吗?

\medskip\noindent
\begin{minipage}{0.5\linewidth}\parindent2em
如\cref{fig:10-22} 所示,把 \qty{1}{kg} 的重物挂在一段尼龙丝的中点。根据力的平行四边形法则,你可以找出 $F$ 和 $G$ 的关系式。

用手拉着尼龙丝的另一端,并沿着箭头 $a$ 所示的方向拉尼龙丝。
当逐渐拉紧尼龙丝时,$\alpha$ 角增大,力 $F$ 也随着增大。
用量角器量自尼龙丝刚刚被扯断时的 $\alpha$ 角,就可以知道尼龙丝的抗断拉力。
\end{minipage}\hfill
\begin{minipage}{0.45\linewidth}\centering
  \begin{figurehere}
    \includegraphics{10-22.pdf}
    \caption{}\label{fig:10-22}
  \end{figurehere}
\end{minipage}


\section{滴水法测重力加速度}
利用水滴下落可以测出重力加速度,调节水龙头,让水一滴一滴地流出。
在水龙头正下方放一个盘子,使水滴落到盘子上。
要把盘子垫起来,以便能清晰地听到水滴碰到盘子的响声。

细心地调整阀门,使第一个水滴碰到盘子的瞬间,第二个水滴正好从阀门处开始下落。
你一边听水滴碰盘子的响声,一边注视着阀门处的水滴,就很容易做到这一点。
这样调整好之后,水滴从阀门落到盘子经过的时间,就正好等于相继滴下的两个水滴之间的时间间隔。

数出在半分钟或一分钟内滴下的水滴的数目,或者测出下落 \numrange{50}{100} 个水滴经过的时间,就可以算出水滴下落的时间 $t$。
用米尺量出水滴下落的距离 $h$。
将 $t$、$h$ 值代入公式 $h=\frac{1}{2}gt^2$ 中,就可以计算出重力加速度 $g$。

\section{用秒表测量玩具手枪子弹射出的速度}
根据你学过的竖直上抛运动的知识,用一只秒表就可以简便地测出玩具手枪子弹射出的速度。

让子弹从枪口竖直向上射出,用秒表测出子弹从射出枪口到落回原地经过的时间 $t$。
设子弹射出的速度为 $v_0$,子弹从射出到落回原地所用的时间
\[t=\frac{2v_0}{g}.\]
由此可以求出子弹射出的速度
\[v_0=\frac{gt}{2}.\]

用这种方法测出玩具手枪子弹射出的速度。

\section{用尺测量玩具手枪子弹射出的速度}
根据你学过的平抛运动的知识,用尺可以简便地测出玩具手枪子弹射出的速度。

让子弹从高度为 $h$ 的地方水平射出,用卷尺量出子弹落地处到射自处的水平距离 $l$ 和高度 $h$。
如果子弹的射出速度 $v_0$,那么,
\[\begin{split}
    h&=\frac{1}{2}gt^2,\\
    l&=v_0t.
\end{split}\]
由此可以求出子弹射出的速度
\[v_0=l \sqrt{\frac{g}{2h}}.\]

用这种方法测出玩具手枪子弹射出的速度。

\section{估测自行车受到的阻力}

骑自行车时,如果停止用力蹬脚踏板,由于受到阻力,自行车在水平路面上前进一段路程就停下来。
设计一个实验,测量自行车在这段路程里所受的平均阻力。
在这个实验里,你要测量些什么?
实际测量一下,你自己或你的同学骑自行车停下来时,在这段路程里受到的平均阻力是多少?

\section{验证向心力公式}
\noindent
\begin{minipage}{0.66\linewidth}\parindent2em
用下面的方法可以验证向心力公式。
如\cref{fig:10-23} 那样,把尼龙绳穿过圆珠笔杆,在绳的两端分别拴上大小不同的两个石块。手握笔杆,抡动小石块,使它做匀速圆周运动,并且使大石块的位置保持基本上不动。
这时使小石块做匀速圆周运动的向心力就等于大石块的重量(想一想,为什么)。
把小石块转动的半径 $r$ 改变三次,测出每次的 $r$ 和每次小石块转动 20 圈所用的时间 $t$。
算出小石块各次转动的角速度 $\omega$,再测出大石块的质量 $M$ 和小石块的质量 $m$。
利用以上测得的数据算出每次小石块做匀速圆周运动所需的向心力 $m\omega^2r$,看看是否都等于大石块的重量 $Mg$。
\end{minipage}\hfill
\begin{minipage}{0.33\linewidth}\centering
\begin{figurehere}
  \includegraphics{10-23.pdf}
  \caption{}\label{fig:10-23}
\end{figurehere}
\end{minipage}

\medskip
要注意,一定要把石块拴牢靠,以免实验时石块飞出,发生意外。

\section{制作杆秤}
学习了物体平衡的知识,你可以自己制作一把杆秤。

取一根 \qtyrange{30}{50}{cm} 的细木棍作秤杆,一个质量 \qty{1}{kg} 左右的物体作秤锤。
照\cref{fig:6-26} 那样先确定秤钩和提纽的位置。
然后在秤钩不挂物体的情况下,把秤锤挂在秤杆上,提起提纽,使秤杆平衡,这时秤锤的位置就是秤的零刻度 $A$ 点(这点也叫定盘星)。
再把质量为 \qty{1}{kg} 的物体挂在秤钩上,调整秤锤的位置,使秤杆平衡,这时秤锤的位置就是秤的 \qty{1}{kg} 刻度点。
再在秤钩上挂质量为 \qty{2}{kg}、 \qty{3}{kg} 的物体,使秤杆平衡,找出 \qty{2}{kg}、\qty{3}{kg} 刻度的位置。
你将发现这几个刻度间的距离是均匀的(为什么,请同学们自己证明)。根据这个规律,你可以在秤杆上找出 \qty{4}{kg}、\qty{5}{kg} 等刻度的位置,把每千克刻度间的距离等分成 10 份,每份间的距离就代表 \qty{0.1}{kg},这样你的杆秤就做成了。

把你制作的杆秤跟商店里层的秤核对一下,看看你的杆秤用起来准不准?

如果要增大杆秤的称量范围,想一想应该怎么办?

\section{研究小球滚下的位置}
如\cref{fig:10-24} 所示,让小球从斜面上某一位置滚下,如果小球在运动中受到的摩擦阻力很小,可以忽略不计,你能否预计出小球落在地面上的位置?
在你预计的位置上放一个塑料杯子,看看小球滚下时是否落入杯中。
这个实验说明了什么问题?
\begin{figure}
  \includegraphics{10-24.pdf}
  \caption{}\label{fig:10-24}
\end{figure}

\chapter*{学生实验}
\addcontentsline{toc}{chapter}{学生实验}
\stepcounter{chapter}
\section{观察磁铁对电流的作用}
这个实验我们先用左手定则来判断磁场对电流的作用力的方向,然后再用实验验证。

如\cref{fig:10-1} 所示,将矩形线圈悬挂在支架上,线圈的 $AB$ 边悬在蹄形磁铁的两极间。
按照\cref{fig:10-1} 把电路连接好。

\medskip\noindent
\begin{minipage}{0.45\linewidth}\parindent2em
根据电键闭合后 $AB$ 边中的电流方向和 $AB$ 边所在处的磁场方向,用左手定则判断出磁场对 $AB$ 边的作用力的方向。
合上电键 $K$,观察 $AB$ 边向哪个方向运动,与你判断的是否一致。
注意:只要看出了 $AB$ 边的运动方向就要立即断开电键,以免电池或线圈烧坏,下面的实验也要这样。
\end{minipage}\hfill
\begin{minipage}{0.5\linewidth}\centering
  \begin{figurehere}
    \includegraphics{10-1.pdf}
    \caption{}\label{fig:10-1}
  \end{figurehere}
\end{minipage}

\medskip
把与电池正、负极连接的两条导线交换一下,改变 $AB$ 边中的电流方向,用左手定则判断出磁场对 $AB$ 边的作用力的方向。
合上电键 $K$,观察 $AB$ 边向哪个方向运动,与你判断的是否一致。$AB$ 边的受力方向与上一次比较改变没有?

把磁铁的 $N$、$S$ 极调换一下,改变磁场的方向,先用左手定则判断出磁场对 $AB$ 边的作用力的方向,然后再做实验,看一看与你判断的是否一致。
$AB$ 边的受力方向与上一次比较改变了没有?

同学们自己设计一个表格,在每次实验前,先将 $AB$ 边中的电流方向以及磁铁的磁场方向填入表中,再根据实验时 $AB$ 边的运动方向,把磁铁对电流的作用力的方向填入表中。

\section{研究电磁感应现象}
关于感生电流产生的条件,我们已经随同老师一起做过实验,现在用\cref{fig:2-3} 所示的装置来研究怎样判断感生电流的方向。为了加强实验效果,这里用的螺线管 $A$ 带有铁芯。

首先查明电流表指针的偏转方向和螺线管中电流方向的关系,然后把蓄电池(或其他低压电源)、开关和螺线管 $A$ 串联成一个电路,把电流表和螺线管 $B$ 串联成另一个电路。

接通电源,给螺线管 $A$ 通电,然后把它插入螺线管 $B$ 中,停留一会儿再取出来。
同时注意,当螺线管 $A$ 在插入或取出时,跟螺线管 $B$ 相连的电流表的偏转方向,并记下 $B$ 中感生电流的方向。

改变螺线管 $A$ 中的电流方向,重做上面的实验,观察跟螺线管 $B$ 相连的电流表的偏转方向,并记下 $B$ 中感生电流的方向。

把螺线管 $A$ 放在螺线管 $B$ 中不动,观察在给螺线管 $A$ 通电和断电的瞬间,跟螺线管 $B$ 相连的电流表的偏转方向,并记下 $B$ 中感生电流的方向。
改变螺线管 $A$ 的通电方向,再观察在通电和断电的瞬间,跟螺线管 $B$ 相连的电流表的偏转方向,并记下 $B$ 中感生电流的方向。

分析你的实验并回答下列问题:
\begin{enumerate}
  \item 螺线管 $B$ 内部的磁场增强时,$B$ 中感生电流的磁场方向怎样?螺线管 $B$ 内部的磁场减弱时,$B$ 中感生电流的磁场方向怎样?
  \item 归纳出判断感生电流方向的规律。
\end{enumerate}

\section{用示波器观察交流电的波形}
这个实验是用示波器来观察交流电的波形。
我们讲过,示波器自己能发出正弦交流信号。
我们就先观察示波器自己的正弦交流信号。
然后再观察从信号源输入的交流信号。

开机前,先把辉度调节旋钮反时针转到底,衰减旋钮置于正弦符号“\tikz \draw[x=.6ex,y=.8ex] (0,0) sin (1,1) cos (2,0) sin (3,-1) cos (4,0)--cycle;”档,Y 增益旋钮顺时针转到底,扫描范围置于 \qtyrange{10}{100}{Hz} 档,其余各个旋钮置于中间位置。
打开电源开关,经预热后,顺时针旋转辉度调节旋钮,可以看到不稳定的不一定清晰的波形。
调整聚焦调节和辅助聚焦旋钮,使图线清晰。
调整垂直位移和水平位移旋钮,使图象位置适中。
调整 X 增益旋钮,使图象大小合适。
作好这些准备工作后,把同步开关置于“$+$”位置,把扫描微调旋钮先顺时针转到底再慢慢地反时针旋转,当转到某一位置时,可以看到一个稳定的完整的正弦波形(\cref{fig:10-2a}),屏上还同时出现的水平亮线是扫描的回扫线。
把同步开关置于“$-$”位置,又可看到波形改变半个周期(\cref{fig:10-2b})。
如果扫描频率恰好是信号频率的 $1/n$($n$ 为整数),将可看到稳定的 $n$ 个完整波形。继续慢慢地反时针旋转扫描微调旋钮,减小扫描频率,就可以看到屏上出现稳定的两个、三个正弦波形。

\begin{figure}
  \begin{minipage}[b]{0.6\linewidth}
    \begin{minipage}[b]{0.5\linewidth}\centering
      \includegraphics{10-2a.pdf}
      \subcaption{}\label{fig:10-2a}
    \end{minipage}%
    \begin{minipage}[b]{0.5\linewidth}\centering
      \includegraphics{10-2b.pdf}
      \subcaption{}\label{fig:10-2b}
    \end{minipage}
    \caption{}\label{fig:10-2}
  \end{minipage}%
  \begin{minipage}[b]{0.4\linewidth}\centering
    \includegraphics{10-3.pdf}
    \caption{J2465型学生信号源的面板}\label{fig:10-3}
  \end{minipage}
\end{figure}

现在再来观察从信号源输入的交流信号。
我们用的信号源是 J2465 型学生信号源,这种信号源可以输出低频、高频正弦交流信号和高频调幅信号(在\cref{chp:electromagnetic_wave}将讲到什么是调幅信号)。
J2465 型学生信号源的面板如\cref{fig:10-3} 所示,下面先介绍它的使用方法。

低频正弦交流信号从右边的两个低频输出接线柱输出。
它们的上边是低频增幅旋钮,顺时针旋转它,低频输出的电压连续增大。
中间的频率选择旋钮用来改变低频输出的频率;它有五档:\qtylist{500;1000;1500;2000;2500}{Hz}。

高频信号从左边的两个高频输出接线柱输出。
它们的上边是高频增幅旋钮,顺时针旋转它,高频输出的电压连续增大。
上边的频率调节旋钮用来连续改变高频输出的频率。
当右边的选择开关在位置“\MyRoman{1}”时,频率改变范围是 \qtyrange{500}{1700}{kHz};在位置“\MyRoman{2}”时,是 \qtyrange{400}{580}{kHz}。

需要高频正弦交流信号时,左边的选择开关应置于“等幅”位置。
当这个选择开关置于“调幅”位置时,从高频输出接线柱输出的是高频调幅信号。
调幅度的大小用低频增幅旋钮来调节。
调制信号的频率用频率选择旋钮来选择。

让我们观察从信号源输入的低频正弦信号。
为此,先用导线把信号源的两个低频输出接线柱跟示波器的“Y 输入”和“地”两个接线柱连接上;把信号源的低频增幅旋钮转到中间位置;把示波器的衰减旋钮从正弦符号档转到最高档。
打开它们的电源开关,示波器的准备工作跟前面相同。
逐步减小衰减档并调整 Y 增益,使图象的竖直幅度适宜。
然后根据输入的信号频率选择适当的扫描范围并调整扫描微调,就可以看到稳定的整数个完整的波形。
再慢慢调整扫描微调,可以看到波形数发生改变。
调整 X 增益,可以看到波形的水平幅度的改变。
把同步开关从“$+$”位置扳到“$-$”位置,可以看到波形改变半个周期。
旋转信号源的低频增幅旋钮,观察波形的竖直幅度的改变。
在观察过程中,必要时应随时调整辉度调节、聚焦调节和辅助聚焦,使图象亮度适宜,图线清晰。

利用示波器还可以测出输入的交流信号的电压的最大值并进而算出它的有效值。
测量时应注意把 Y 增益旋钮顺时针旋转到底,还应根据衰减乘以相应的倍数。
你在实验中,当低频增幅旋钮转到最大时,信号电压的最大值和有效值各是多少?

改变输入信号的频率,再次进行各项调整和观察。

如果有多余的时间,还可以继续观察高频正弦信号和高频调幅信号。
各项调整方法跟观察低频正弦信号时基本相同。

\section{用示波器观察交流电的整流和滤波}\label{sec:rectifier_filter}

这个实验通过示波器观察波形来了解整流滤波电路的作用。

\cref{fig:10-4} 是一个带 $\pi$ 型滤波器的半波整流路。
交流电源的电压,滤波电容器 $C_1$、$C_2$ 的电容,电阻 $R$ 的阻值,都是根据负载 $R_{\text{负}}$ 的要求选定的。
$C_1$、$C_2$ 越大,滤波效果越好。
$R$ 大些,滤波效果好些,但电压损失也大些。
图中的数据可作参考。
这个电路可提供 \qtyrange{5}{6}{V} 的直流电。
\begin{figure}
  \includegraphics{10-4.pdf}
  \caption{}\label{fig:10-4}
\end{figure}

在一张适当大小的白纸上,画出图中所示的电路,并标出各个元件的规格。
把这张电路图平铺在一块铁板上。
选取你需要的实验元件(\cref{fig:10-5} 所示的是专为实验用的几种元件)。
将各个元件放在电路图标出的位置上(应注意二极管和电解电容器的正负极不要接错)。
用导线将各元件按照电路图连接起来。
为了观察整流前后的波形,$P$ 点先不要接通。

\begin{figure}
  \begin{minipage}{0.23\linewidth}\centering
    \includegraphics{10-5a.pdf}
    \subcaption{晶体三极管}
  \end{minipage}
  \begin{minipage}{0.23\linewidth}\centering
    \includegraphics{10-5b.pdf}
    \subcaption{晶体二极管}
  \end{minipage}
  \begin{minipage}{0.23\linewidth}\centering
    \includegraphics{10-5c.pdf}
    \subcaption{电容}
  \end{minipage}
  \begin{minipage}{0.23\linewidth}\centering
    \includegraphics{10-5d.pdf}
    \subcaption{电阻}
  \end{minipage}
  {\par\footnotesize 元件装在透明塑料盒内,盖上画着元件的符号,并有弹簧式的接
线头,盒底有块磁铁,使元件能平稳地安放在实验用的铁板上。}
  \caption{实验用的元件}\label{fig:10-5}
\end{figure}

电路检查无误后,即可接上 \qty{6}{V} 的交流电源,用示波器观察波形。
先把负载 $R_{\text{负}}$ 改接在 $A$ 点和“地”(即 $BB'$ 导线)之间,把 $A$ 点接示波器的 Y 输入,$BB'$ 导线接示波器的“地”,观察整流前的交流电压波形。
再把负载 $R_{\text{负}}$ 改接在 $A$ 点和“地”之间,把 $A'$ 点接示波器的 Y 输入,$BB'$ 导线仍然接示波器的“地”,观察整流后的电压波形。
然后接通 $P$ 点,把负载 $R_{\text{负}}$ 接在\cref{fig:10-4} 所示的位置,把 $A$ 点接示波器的 Y 输入,$BB'$ 导线仍然接示波器的“地”,观察滤波后的电压波形。
在观察波形时,在坐标纸上把三种波形记录下来,以便比较。

有兴趣的同学还可以改变电容器 $C_1$、$C_2$ 和电阻 $R$ 的数值,观察 $A''$ 点对“地”的电压波形的变化,研究它们对滤波效果的影响。

\section{研究变压器的作用}
这个实验利用变压器模型来研究变压器的作用。

这个变压器模型有三个线圈,线圈 \MyRoman{1} 为 120 匝,线圈 \MyRoman{2} 为 240 匝,线圈 \MyRoman{3} 为 60 匝。

现在拿线圈 \MyRoman{1} 作原线圈,线圈 \MyRoman{3} 作副线圈,把它们套在一起,将硅钢片插入原、副线圈中,这样就装成了一个降压变压器。

记下原线圈的匝数 $n_1$ 和副线圈的匝数 $n_2$。
将原线圈的两端接低压交流电源,副线圈的两端跟小电灯泡相连。
用交流电压表分别测量原、副线圈两端的电压 $U_1$ 和 $U_2$,记下测得的数据。
看看它们的电压比 $U_1/U_2$ 跟匝数比 $n_1/n_2$ 有什么关系。

用交流电流表分别测出原、副线圈中的电流 $I_1$ 和 $I_2$,记下测得的数据。
看看原、副线圈中的电流比 $I_1/I_2$ 跟它们的匝数比 $n_1/n_2$ 有什么关系。

根据测得的原、副线圈的电压和电流的数值,算出变压器的输入功率 $U_1I_1$ 和输出功率 $U_2I_2$。
你这个变压器的效率是多少?

在输出电路中再并联几个小灯泡。
负载增加后,副线圈中的电流增大了,原线圈中的电流怎样变化?
输出功率和输入功率怎样变化?

拆开变压器,换上线圈 \MyRoman{2} 作副线圈,然后再将硅钢片插入,装成一个升压变压器。
按上面的步骤重做一遍实验。

根据你的实验,回答下面的问题:
\begin{enumerate}
  \item 变压器原、副线圈两端的电压跟它们的匝数有什么关系?
  \item 变压器原、副线圈中的电流跟它们的匝数有什么关系?
  \item 变压器的输出电流改变时,输入电将怎样改变?输出功率改变时,输入功率将怎样改变?输出功率和输入功率是否相等?为什么?
\end{enumerate}

\section{安装简单的收音机}
我们用方框图已经讲过简单晶体管收音机的工作原理。
现在来连接这种收音机的电路,学习它的安装和调试方法。

\cref{fig:10-6} 是一个两管收音机的线路图,它包括调谐、高频放大、检波、低频放大四个部分,图中已经用虚线隔开。
图中的 $R_1^*$ 和 $R_2^*$ 叫做偏流电阻。
\begin{figure}
  \includegraphics{10-6.pdf}
  \caption{简单收音机的线路图}\label{fig:10-6}
\end{figure}

按照\cref{fig:10-6} 连接电路,连接的方法跟实验\ref{sec:rectifier_filter}相同。
先在白纸上画出电路图,然后根据规格选取需要的元件,放在纸上摆好,再用导线把它们连接起来。
偏流电阻 $R_1^*$ 和 $R_2^*$ 的阻值待定,可以先空着,等调试后再连入。

连接完毕,要按照电路图全面仔细检查一遍,确认无误后,即可接上电源进行调试。

对本机的调试,就是调整两个三极管的偏流电阻。把毫安表或万用电表的毫安档接入三极管的集电极电路(图中画“$\times$”号处),用 \qty{20}{k\ohm} 的固定电阻和 \qty{470}{k\ohm} 的电位器串联起来暂代偏流电阻连入电路。
调节电位器,使高频三极管集电极电流为 \qty{1}{\micro A} 左右,低频三极管集电极电流为 \qty{2}{\micro A} 左右。
然后取下串联的固定电阻和电位器,用万用电表分别测出它们的总电阻,用阻值相同的固定电阻连入电路,调试工作就完成了。

调试好以后,即可调节可变电容器来试听电台的广播。
本机的灵敏度不高,在广播电台较远时可能收听不到。
遇到这种情况,可以用信号源发出高频调幅信号来代替广播电台,这时从耳机里可以听到嗡嗡的调制信号的音频交流声。

\section{测定玻璃的折射率}

在这个实验里,我们用两面平行的玻璃砖来测定玻璃的折射率。

从\cpageref{exp:5-1}\cref{exp:5-1}可知,当光线斜射入两面平行的玻璃板时,从玻璃板射出的光线传播方向不变,出射光线跟入射光线相比,只有一定的侧移。
只要我们找出跟某一入射光线对应的出射光线,就能求出在玻璃中对应的折射光线,从而求出折射角,再根据折射定律,就可以求出玻璃的折射率。

\medskip\noindent
\begin{minipage}{0.5\linewidth}\parindent2em
实验的具体做法如下:照\cref{fig:10-7} 那样,先在一张白纸上画直线 $aa'$ 作为玻璃砖的一个界面,过 $aa'$ 上的一点 $O$ 画界面的法线 $NN'$,再画直线 $AO$ 作为入射光线。
把长方形玻璃砖放在白纸上,使它的长边跟 $aa'$ 对齐,画出玻璃砖的另外一边 $bb'$。
在直线 $AO$ 上竖直插上两枚大头针 $P_1$、$P_2$,透过玻璃砖观察大头针 $P_1$、$P_2$ 的像。
移动视线的方向,直到 $P_1$ 的像被 $P_2$ 的像挡住。
再在观察的这一侧插两枚大头针 $P_3$、$P_4$,使 $P_3$ 挡住 $P_1$、$P_2$ 的像,$P_4$ 挡住$P_3$ 和 $P_1$、$P_2$ 的像。
\end{minipage}\hfill
\begin{minipage}{0.45\linewidth}\centering
  \begin{figurehere}
    \includegraphics{10-7.pdf}
    \caption{}\label{fig:10-7}
  \end{figurehere}
\end{minipage}

\medskip
记下 $P_3$、$P_4$ 的位置。
移去玻璃砖和大头针。
过 $P_3$、$P_4$ 引直线 $EB$,与 $bb'$ 交于 $E$,$EB$ 就表示沿 $AO$ 方向入射的光线通过玻璃砖后传播的方向。
连接 $OE$,$OE$ 就是玻璃砖内折射光线的方向。
入射角 $i=\angle AON$,折射角 $r=\angle EON'$。

用量角器量出入射角和折射角,从三角函数表中查出它们的正弦值,把这些数据记在自己设计的表格里。

用上面的方法分别求出入射角是 \ang{30}、\ang{45}、\ang{60} 时的折射角,查出它们的正弦值,把得到的数据记在表格里。

算出不同入射角时 $\sin i/\sin r$ 的比值,比较一下,看它们是否接近于一个常数。
求出几次测得的 $\sin i/\sin r$ 的平均值,作为测得的玻璃的折射率 $n$。

\section{测定凸透镜的焦距}
测定凸透镜的焦距有各种不同的方法,在这个实验里我们用三种比较简单的方法来测定。

\subsection{平行光聚焦法}

平行于凸透镜主轴的光线,经凸透镜折射后将会聚于焦点,利用凸透镜的这一特性,可以测出它的焦距。
方法是把凸透镜对着远处的光源(例如太阳),在透镜的另一侧放一个光屏(或一张白纸),调节透镜和白纸间的距离,直到屏上出现的光斑最亮最小为止。
这个光斑就是透镜的焦点。
用直尺量出这时透镜到光斑间的距离,就得到凸透镜的焦距。

\subsection{利用透镜成像公式}

把点燃的蜡烛、凸透镜、光屏照\cref{fig:5-38} 那样放在光具座上,调整它们的高度,使烛焰和光屏的中心位于凸透镜的主轴上(共轴)。
调节蜡烛和光屏到透镜的距离,使光屏上呈现出烛焰的清晰的像,量出这时的物距和像距,填入自己设计的表格中。

改变蜡烛到凸透镜的距离,按照上段的要求再做两次。
把测得的数据也填入表格中。

根据测得的三组数据,利用凸透镜的成像公式
\[\frac{1}{u}+\frac{1}{v}=\frac{1}{f}\]
算出三次得到的 $f$ 值,求出它们的平均值,作为测得的凸透镜的焦距。

\subsection{利用公式 \texorpdfstring{$f=\dfrac{L^2-d^2}{4L}$}{f=(L2-d2)/4L}}

从\cpageref{exp:5-2}\cref{exp:5-2}知道,如果保持物体和光屏之间的距离 $L$ 不变,在物体和光屏之间移动凸透镜,使物体在光屏上先后两次成像,测出凸透镜的两个位置间的距离 $d$,那么凸透镜的焦距
\[f=\dfrac{L^2-d^2}{4L}.\]
需要注意的是,用这种方法测焦距时,一定要使 $L>4f$,才能在光屏上得到物体的像。
(有兴趣的同学可以自己考虑并证明一下这个问题。提示:利用关系 $L=u+v$,从 $\dfrac{1}{u}+\dfrac{1}{v}=\dfrac{1}{f}$中解出 $v$,可得只有在 $L \geqslant 4f$ 时,$v$ 才有实数解)

实验时,可先用平行光聚焦法粗测出凸透镜的焦距 $f$,然后再利用\cref{fig:5-38} 中的装置来做。
使蜡烛到光屏的距离 $L>4f$,把凸透镜从蜡烛附近逐渐向光屏移动,同时注意观察光屏上烛焰的像。
当第一次出现清晰的像时,在光具座上记下凸透镜的位置 1。
继续向光屏移动凸透镜,当光屏上第二次出现烛焰的清晰的像时,在光具座上记下凸透镜的位置 2。
量出 1、2 两个位置间的距离 $d$,把测得的数据记录在自己设的表格中。

改变 $L$,再重做两次,记录下测得的数据。
用公式 $f=\dfrac{L^2-d^2}{4L}$ 求出凸透镜的焦距。
算出三次求得的 $f$ 的平均值,作为凸透镜的焦距。

\section{组成显微镜模型}
在这个实验里,我们用两个焦距为 $f_1$ 和 $f_2$ 的凸透镜在光具座上组成显微镜模型。

实验装置如\cref{fig:10-8} 所示。
把一块玻璃竖立在光具座的一端,玻璃上粘一个小物体(例如一个小昆虫等)。
用焦距 $f_1$ 较短的凸透镜 $L_1$ 做物镜,焦距 $f_2$ 较长的凸透镜 $L_2$ 做目镜,把它们安装在光具座上,在 $L_1$ 和 $L_2$ 之间放一个光屏。
调整 $L_1$、$L_2$ 的高度,使它们共轴。
同时调整玻璃和光屏的高度,使玻璃上的小物体和光屏的中心在 $L_1$ 和 $L_2$ 的主轴上。
\begin{figure}
  \includegraphics{10-8.pdf}
  \caption{}\label{fig:10-8}
\end{figure}

用手电筒(或其他光源)照亮玻璃上的小物体,移动物镜 $L_1$,使它到物体的距离 $2f_1>u_1>f_1$,同时移动光屏,直到在光屏上出现小物体清晰的倒立放大的实像。
移动目镜 $L_2$,同时通过 $L_2$ 观察光屏上的像,待看到清晰的放大虚像时,移去光屏,再通过 $L_2$ 观察,这时看到的放大虚像跟有光屏时是一样的。
这表明这个虚像确实是以 $L_1$ 所成的实像为物而产生的。

\section{利用双缝干涉测定光波的波长}
这个实验是利用双缝干涉条纹来测定单色光的波长。
实验装置如\cref{fig:10-9} 所示,光源发出的光经滤光片成为单色光。
单色光通过单缝后,经双缝产生干涉。
干涉条纹可从屏上观察到。
\begin{figure}
  \includegraphics{10-9.pdf}
  \caption{}\label{fig:10-9}
\end{figure}

双缝间的距离 $d$ 是已知的,双缝到屏的距离 $l$ 和相邻两条干涉条纹间的距离 $\Delta x$ 能够测量,因此可以利用下面的公式求出波长
\[\lambda=\frac{d\Delta x}{l}.\]

现在来做实验。
把直径约 \qty{10}{cm}、长约 \qty{1}{m} 的遮光筒水平放在光具座上,筒的一端装有双缝,另一端装有毛玻璃屏。
先取下双缝,打开光源,调节它的高度,使它发出的一束光能够沿着遮光筒的轴线把屏照亮。
然后放好单缝和双缝,单缝和双缝间的距离为 \qtyrange{5}{10}{cm},使缝互相平行,中心大致位于遮光筒的轴线上。
这时,在屏上就会看到双缝的干涉条纹,你看到的干涉条纹是不是彩色的?

放上滤光片,注意观察亮条纹间的距离是否相等。
测出 $n$ 条干涉条纹间的距离 $a$,那么相邻两条干涉条纹间的距离
\[\Delta x=\frac{a}{n-1},\]
再用米尺测出双缝到屏的距离 $l$。
换用另外颜色的滤光片,观察干涉条纹间的距离有什么变化。

根据已知的双缝间的距离 $d$ 和实验所得的数据,代入公式求出单色光的波长。

你求出的单色光的波长是多少?
利用书中所附的连续光谱图查出这种波长的光的颜色,跟滤光片的颜色相比较,看着它们是否一致?

\section{观察光的衍射现象}
\subsection{观察单缝衍射}
用一个具有直长灯丝的白炽灯泡作线状光源(用距离较远的日光灯管也可以)。
调节游标卡尺两脚间的距离,形成一个 \qty{0.5}{mm} 宽的狭缝。
在距灯丝几米远处,使卡尺的狭缝与直灯丝平行,眼睛紧靠狭缝,通过狭缝观察线光源。
这时可以看到光通过狭缝后产生的衍射现象:光束变宽,并有许多条彩色条纹。

调节狭缝的宽度,使它变宽,观察衍射条纹有什么变化;使狭缝变窄,再观察衍射条纹有什么变化。

\subsection{观察小孔衍射}
用手电筒的小灯泡作点光源,在铝箔(或胶片)上打出尺寸不同的小孔。
在距点光源 \qtyrange{1}{2}{m} 处,使眼睛紧靠小孔,观察光通过小孔的衍射现象,可以看到彩色的圆环,孔越小,圆环面积越大。

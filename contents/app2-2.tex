\chapter*{课外实验活动}
\addcontentsline{toc}{chapter}{课外实验活动}
\setcounter{section}{0}
\section{观察扩散现象}
在容器里装一半水,然后用长颈漏斗小心地把硫酸铜溶液倒进容器的底部。
硫酸铜溶液比水的密度大,沉在容器下部。
可以看到无色的水和蓝色的硫酸铜溶液的界面非常清楚。
由于扩散,经过几天以后,界面上方的水变蓝,界面下方的溶液的颜色变浅,界面变得模糊了。
经过更长的时间,全部溶液将变成均匀一致的颜色。
把装有水和硫酸铜溶液的容器放在不容易碰到的地方。
每隔一天,按时观察现象,看看要多长时间全部液体才变成均匀一致的颜色。

\section{自制冰淇淋}
冰和盐的混合物的熔点可以降到零下二十多摄氏度,因此可以用冰盐混合物进行冷却,现在我们用冰盐混合物致冷来自制冰淇淋。

制冰淇淋需要的原料有牛奶、鸡蛋、玉米粉、糖以及其他香料。把配好的原料\footnote{原料的配法:把生鸡蛋打开放在碗里,加入少许玉米粉搅匀。然后慢慢冲入煮沸的加糖牛奶并且要边冲边搅。最后,放入香料,再用锅稍稍熬一会儿,冷却后即成。}放在小铝锅里。找一个大容器(如一只大锅),里面装冰盐混合物(大致是三份冰,一份盐的比例)或雪盐混合物。
把小铝锅放在冰盐混合物中,只有锅盖露在外面。
过一些时间打开锅盖,看看是否快要冻结。
当快要冻结时要搅拌,否则结成的冰粒很粗,不好吃;也不要冻结得太厉害,因为冻得太硬,也不好吃。

\section{人造云雾}
用冰盘混合物致冷可以人工造成云雾。
找一个小铁罐(如一个罐头盒子),放在冰盐混合物或雪盐混合物中,小铁罐里的空气很快就冷却。
对着小铁罐吹口气,水蒸气就被带进铁罐里。
由于里面温度很低,水蒸气凝结成小水滴,这样就造出了淡淡的云雾。
用手电筒照一下小铁罐里面,就可以看到人造的云雾。

\section{测定水的汽化热}
用下述办法可以粗略地测定水的汽化热。
在铝锅里装一些室温($\theta_1\unit{\celsius}$)下的水,将铝锅放在炉子上加热。
设经过时间 $t_1$ 水达到沸点 $\theta_2\unit{\celsius}$,再经过时间 $t_2$ 水全部汽化完。
已知水的比热 $c=\qty{4.2e3}{J/(kg.\celsius)}$,由下式即可求出水的汽化热:
\[L=c(\theta_2-\theta_1)\frac{t_2}{t_1} \,\unit{J/kg}.\]

上式是这样得到的:设锅里的水每秒吸收的热量为 $g$\,\unit{J},把室温的水加热到沸点所需的热量 $Q_1$ 和水全部汽化所需的热量 $Q_2$ 分别为
\[\begin{split}
    Q_1&=mc(\theta_2-\theta_1)=gt_1,\\
    Q_2&=mL=gt_2.
\end{split}\]
将上面两式相除,得到
\[\frac{c(\theta_2-\theta_1)}{L}=\frac{t_2}{t_1}.\]
由此即可求得 $L$。

用这个办法只能粗略地测定水的汽化热。误差的主要来源是什么?

\section{估计水升高的温度}
有一个直径约为 \qty{1}{cm} 的小玻璃管,装入 \qty{10}{ml} 室温的水。
点燃一根火柴,放在小玻璃管底部给水加热。
待这根火柴燃烧到尽头时,你能估计出小玻璃管里的水上升了多少摄氏度吗?
例如是上升百分之几摄氏度,还是十分之几摄氏度、几摄氏度、十几摄氏度、二十几摄氏度……。
你估计的根据是什么?做一做看,水的温度实际上升了多少摄氏度?跟你估计的是否一致?不一致的原因在哪里?

\section{用自制的验电器做静电实验}
\medskip\noindent
\begin{minipage}{0.65\linewidth}\parindent2em
  如\cref{fig:9-15} 所示,是一个简易验电器。
  金属丝对折后穿过瓶盖插入透明的玻璃瓶里,取两条约长 \qty{2}{cm}、宽 \qty{4}{mm} 的金属箔,分别挂在金属丝的两端。
  两金属箔不带电时自由下垂,带电时互相推斥而张开。

  照上面说的那样,自己做一个验电器。
  要注意,应把瓶盖擦干净,不能潮湿,以保证它有良好的绝缘性能。
  利用你自制的验电器,以及塑料尺子、金属小筒、金属网等器材,做一做摩擦起电、感应起电、导体上的电荷分布、静电屏蔽等实验。
\end{minipage}\hfill
\begin{minipage}{0.3\linewidth}\centering
  \begin{figurehere}
    \includegraphics{9-15.pdf}
    \caption{}\label{fig:9-15}
  \end{figurehere}
\end{minipage}

\section{自制电池}
把不同的金属放在电解质溶液中,由于化学作用,产生了电动势,就成为一个化学电源。
现在把同样大小的四、五个光洁的锌片和钢片交替相叠,并在锌片和钢片之间夹一片吸满盐水的吸水纸,这样就做成了一个简单的电池。
锌片和锌片相连是电池的负极,铜片和铜片相连是电池的正极。

用细漆包线在指南针外壳上绕 50 圈,当漆包线中有电流通过时,磁针发生偏转,这样用指南针就可以检验有无电流。

把漆包线两端的漆刮掉,露出铜丝,分别接在自制电池组的两极上,检验导线中有无电流通过。

\section{研究电灯泡的电阻}
取一只电灯泡,看一看它上面标明的额定功率和额定电压是多少,并利用额定功率和额定电压的数值算出灯丝的电阻。
你计算得出的电阻是多少?

再用万用表直接测出这只电灯泡灯丝的电阻,你测得的电阻又是多少?
跟计算得出的数值是否一致?
想想看,这是什么道理?
\chapter*{学生实验}
\addcontentsline{toc}{chapter}{学生实验}
\stepcounter{chapter}
\section{验证玻意耳—马略特定律}

在这个实验中,我们用一个带有刻度的注射器近似地验证玻意耳—马略特定律。

实验研究的对象是封闭在注射器里的空气柱。
空气柱的体积可由注射器上的刻度直接读出。
空气柱的压强 $p=p_0\pm \frac{F}{S}$,其中 $p_0$ 为大气压强,$F$ 为活塞对空气柱的压力或拉力,$S$ 为活塞的横截面积。(考虑一下,哪种情况取正号,哪种情况取负号。)

实验时,先用弹簧秤称出活塞和框架的重量。
用刻度尺测出注射器的全部刻度的长度,用这个长度去除它的容积即得活塞的横截面积 $S$。记下气压计(全班共用一个)指示的大气压强 $p_0$。

将活塞插入注射器内一部分后,将注射器的小孔堵住,以封入一定质量的空气。

把注射器固定好。
在框架的两侧加挂钩码,使空气柱的体积减小(\cref{fig:9-1})。改变钩码的个数,再做两次。
记下每次加挂的钩码数和相应的空气柱的体积。
\begin{figure}
  \begin{minipage}[b]{0.48\linewidth}
  \centering
  \includegraphics{9-1.pdf}
  \caption{}\label{fig:9-1}
  \end{minipage}
  \begin{minipage}[b]{0.48\linewidth}
  \centering
  \includegraphics{9-2.pdf}
  \caption{}\label{fig:9-2}
  \end{minipage}
\end{figure}

然后,用弹簧秤钩住活塞框架的上边,慢慢竖直提起活塞,使空气柱的体积增大(\cref{fig:9-2})。
记下三组每提到一定高度时弹簧秤的读数和相应的空气柱的体积。

把记录的数据填入自己设计的表格里。
根据公式 $p=p_0\pm \frac{F}{S}$ 算出各个压强值(要注意活塞和框架的重量对压强的影响)。
求出各个压强 $p$ 跟相应的体积 $V$ 的乘积。
比较这些乘积,能得出什么结论?

\section{验证气体状态方程}
这个实验,我们利用实验一的装置来验证气体状态方程。

跟实验一一样,先用弹簧秤称出活塞和框架的重量;测出注射器的全部刻度的长度,求出活塞的横截面积 $S$;记下这时的大气压强 $p_0$。
在注射器内封入一定质量的空气。

\medskip\noindent
\begin{minipage}{0.7\linewidth}\parindent2em
照\cref{fig:9-3} 那样,固定好注射器和烧杯。
在活塞框架的两侧加挂钩码,用公式 $p=p_0+ \frac{F}{S}$ 计算出空气柱的压强(注意压力 $F$ 中应包括活塞和框架的重量)。
向烧杯里倒入适量的水,使注射器内的空气柱位于水面之下。
经过两分钟左右,用温度计测出水的温度 $t$,可以认为这个温度就是空气柱的温度,把它换算成热力学温度 $T$。
记下这时空气柱的体积 $V$。

改变加挂的钩码数和烧杯中水的温度,再分别做四次上面的实验。
\end{minipage}\hfill
\begin{minipage}{0.25\linewidth}\centering
  \begin{figurehere}
    \includegraphics{9-3.pdf}
    \caption{}\label{fig:9-3}
  \end{figurehere}
\end{minipage}

\medskip
把上面得到的数据填入自己设计的表格里,并算出每次实验得到的 $pV/T$ 的值,看看它们是否相等,从实验可以得出什么结论?

\section{测定冰的熔解热}

单位质量的某种物质熔解成同温度的液体时吸收的热量,叫做这种物质的熔解热。
在这个实验里我们利用量热器用混合法来测定冰的熔解热。

设有 $m_{\text{冰}}\,\unit{g}$ 的冰,温度为 \qty{0}{\celsius},与 $m_{\text{水}}\,\unit{g}$ 温度为 $t_0$ 的水混合,冰全部熔解跟水混合以后的平衡温度为 $t$。
$m_{\text{冰}}\,\unit{g}$ 冰熔解成水并升高到温度 $t$ 吸收的热量,等于 $m_{\text{水}}\,\unit{g}$ 水以及盛水容器从温度 $t_0$ 下降到温度 $t$ 放出的热量,即
\[ m_{\text{冰}}\lambda+m_{\text{冰}}c_{\text{水}}t=(m_{\text{水}}c_{\text{水}}+m_{\text{筒}}c_{\text{筒}})(t_0-t).\]
其中,$\lambda$ 为冰的熔解热,$c_{\text{水}}$ 为水的比热,$c_{\text{筒}}$ 为容器的比热,$m_{\text{筒}}$ 为容器的质量。
这样,把 $c_{\text{水}}$ 和 $c_{\text{筒}}$ 作为已知量,$m_{\text{冰}}$、$m_{\text{水}}$、$m_{\text{筒}}$ 和 $t_0$、$t$ 都可以由实验获得,从而利用上式求出冰的熔解热 $\lambda$。

实验开始时,先用天平称出量热器小筒的质量 $m_{\text{筒}}$(包括搅拌器)。
然后把比室温高 \qtyrange{10}{15}{\celsius} 的温水(\qty{150}{g} 左右)倒入量热器小筒,再称出水和小筒的质量,算出水的质量 $m_{\text{水}}$。
把装着水的量热器小筒放在大筒的木架上,用温度计测出水和量热器小筒的初温 $t_0$。
把准备好的温度为 \qty{0}{\celsius} 的冰块\footnote{实验室里,冰水混合物的温度可以认为是\qty{0}{\celsius}。}(\qty{20}{g} 左右)迅速放入小筒的水中,并盖好量热器盖子。
搅动小筒中的水,同时观察插入量热器里的温度计。
当温度下降到最低时,记录下来的温度 $t$ 就是冰、水混合后的平衡温度。
最后再称量一下量热器小筒和水的质量(其中包括冰的质量),算出冰的质量 $m_{\text{冰}}$。

把由实验得到的数据代入第二段中的公式,求出冰的熔解热。
水的比热 $c_{\text{水}}$,可取为 \qty{4.2e3}{J/(kg.\celsius)},铝制小筒的比热 $c_{\text{筒}}$ 可取为 \qty{8.9e2}{J/(kg.\celsius)},铜制小筒的比热 $c_{\text{筒}}$ 可取为 \qty{3.9e2}{J/(kg.\celsius)}。

实验中要注意读准温度计的示数。
冰块不宜太大,为什么?
在这个实验中,误差的主要来源是什么?

\section{测定空气的相对湿度}
这个实验我们学习测定空气的相对湿度。
实验装置如\cref{fig:9-4} 所示。
圆柱形金属盒的一个底面十分光亮,侧面有开口,开口旁边有一小孔,用来插入温度计。
环形金属片套在金属盒上,它的一面也是十分光亮,井与金属盒的光亮面在同一平面内,金属盒和环形片用胶木垫圈隔开,防止相互间的热传导。
搅拌器插在开口中。

\medskip\noindent
\begin{minipage}{0.6\linewidth}\parindent2em
实验时,先记录下实验室的温度,用柔软的绒布仔细地把金属盒和环形片的光亮面擦得十分干净。
在金属盒里注入约半盒室温的水,再向水里投入适量的碎冰块(注意不要沾污光亮面)。
装上温度计,并使它的刻度向着金属盒的光亮面。
搅动冰块,使水温迅速下降,同时密切注视金属盒和环形片的光亮面,当金属盒的光亮面上刚刚出现细小的露滴时,记录下这一瞬间盒里水的温度。
等水的温度又开始上升,金属盒光亮面上的细小的露滴完全消失时,再记录下这一瞬间的温度,两次温度的平均值就是露点。
\end{minipage}\hfill
\begin{minipage}{0.35\linewidth}\centering
  \begin{figurehere}
    \includegraphics{9-4.pdf}
    \caption{}\label{fig:9-4}
  \end{figurehere}
\end{minipage}

\medskip
从课本里查出温度为测得的露点时水的饱和汽压 $p$,这就是测量时空气中水蒸气的压强,即空气的绝对湿度。
再查出测量时的室温下水的饱和汽压 $P$。
此时的相对湿度就是
\[B=\frac{p}{P}\times 100\%.\]

\section{电场中等势线的描绘}
在这个实验里,我们学习用描迹法画出电场中平面上的等势线。

如\cref{fig:9-5} 所示,在平整的木板上铺一张白纸,白纸上依次铺放复写纸和导电纸,导电纸有导电物质的一面向上。
白纸、复写纸和导电纸一起用图钉固定在木板上。
导电纸上平放着跟它接触良好的两个圆柱形电极,电极 $A$ 与电源的正极相连作为正电荷,电极 $B$ 与电源的负极相连作为负电荷\footnote{我们在\cref{chp:electric_field}学习的是静电场。直接接绘静电场中的等势线是相当困难的,由于静电场和稳恒电流场遵守的规律相似,这里是用在导电纸上形成的稳恒电流场模拟静电场来做实验。}。
两电极之间的距离约为 \qty{10}{cm},电压约为 \qty{6}{V}。
\begin{figure}
  \includegraphics{9-5.pdf}
  \caption{}\label{fig:9-5}
\end{figure}

现在,我们来描绘正、负电荷 $A$、$B$ 在纸面上的等势线。
从灵敏电流表的两个接线柱引出两个探针,用来探测导电纸上的等势点。
先在导电纸平面两电极的连线上,选取间距大致相等的五个点作为基准点,并用探针把它们的位置复印在白纸上。
在某一基准点将一个探针跟导电纸相接触,然后在导电纸平面上两电极连线的一侧,距此基准点约 \qty{1}{cm} 处再选一个点,在此点将另一探针跟导电纸相接触。
一般这时会看到电流表的指针有偏转。
左右移动另一探针的位置,直到找到一点,使电流表的指针没有偏转为止。
电流表的指针没有偏转,说明这个点跟基准点的电势相等。
用探针把这个点的位置复印在白纸上。
照上述方法,在这个基准点的两侧,各探测出五个等势点,每个等势点大约相距 \qty{1}{cm}。
用同样的方法,探测出另外四个基准点的等势点。
最后,取出白纸,根据五组等势点画出五条平滑的曲线,它们就是等势线。
你能不能根据这些等势线在白纸上画出两个异种电荷的电力线:画一画看。

\section{利用电容器放电测电容}
现在,我们通过实验来学习一种测量电容器电容的简单方法。

我们知道,电容器的电容等于电容器所带电量跟两极之间的电势差的比值,即 $C=Q/U$。
因此,如果测量出某一电压下电容器所带的电量,就可以求出电容器的电容。
怎样才能测量出电容器所带的电量呢?
\begin{figure}
  \includegraphics{9-6.pdf}    
  \caption{}\label{fig:9-6}
\end{figure}

测量电路如\cref{fig:9-6} 所示,合上电键 $K'$、$K$,对电容器 $C$ 充电。
当电容器两端电压 $U_c$ 上升到某一稳定电压 $U$ 时,充电完毕。
然后将电键 $K$ 打开,这时容器通过电阻 $R$ 放电,放电电流 $i_c$ 随时间 $t$ 的增加而逐渐减小,放电完毕时 $i_c=0$。
在电容器放电过程中,如果在某一时刻的放电电流为 $i_c$,那么在一小段时间间隔 $\Delta t$ 里,从电容器正极转移到负极上的电量就等于 $i_c\Delta t$,将整个放电过程中每小段时间所转移的电量加起来,就得到电容器所带的电量 $Q$。

按\cref{fig:9-6} 接好线路,电源可用学生电源,电容器 $C$ 可选用 \qty{470}{\micro F} 的电解电容器,微安表可选用 \qty{500}{\micro A} 量程的,$R$ 用 \qty{27}{k\ohm} 的定值电阻,接线时要注意电解电容器的极性不要接反。
接通电源后,先合上电键 $K'$,调节变阻器 $R'$ 使伏特表指示到实验需要的电压值 \qty{12}{V},然后合上电键 $K$,给电容器充电,充电完毕,记下这时伏特表和微安表的读数。
把电键 $K$ 打开,同时开始计时,并且每间隔 \qty{5}{s} 读取一次微安表的电流值,直到电流值减至零为止。

根据记录的数据,在坐标纸上,以时间 $t$ 为横坐标,以电流 $i_c$ 为纵坐标作出 $i_c$--$t$ 图象,然后再根据所画的 $i_c$--$t$ 图象,求出电容器所带电量 $Q$(同学们思考一下,怎样利用 $i_c$--$t$ 图象求出电量 $Q$),最后计算出电容器的电容。

\section{测定金属的电阻率}
这个实验是测定金属的电阻率。

电阻定律告诉我们,用电阻率为 $\rho$ 制成的长 $l\,\unit{m}$、横截面积 $S\,\unit{m^2}$ 的导线的电阻
\[R=\rho\frac{l}{S}.\]

因此,测出一段导线的长度和直径(由直径可以算出横截面积)以及这段导线的电阻,就可以求出制成这段导线的材料的电阻率。

现在有一段长约 \qty{0.5}{m}、直径约 \qty{0.3}{mm},阻值约 \qty{3}{\ohm} 的金属导线,你应当选用哪些实验器材来测定它的电阻率?
考虑一下,这个实验应当怎样进行?
通过实验,你测得制成这段导线的材料的电阻率是多少?

需要注意的是,在给导线通电时,电流不宜太大,想想看,这是为什么?

\section{把电流表改装为伏特表}
我们学习了把电流表改装为安培表和伏特表的原理,在这个实验里,我们练习把电流表改装为伏特表。

改装电流表,需要知道它的三个数据:满度电流 $I_g$,满度电压 $U_g$(电流表的指针偏转到满刻度时加在表头上的电压)和内电阻 $r_g$。
这三个数据中,知道任何两个,就可以根据欧姆定律算出第三个,电流表的 $I_g$ 可以从刻度盘上直接读出,$r_g$ 可用实验方法测出,于是就可以算出 $U_g$。
\begin{figure}
  \includegraphics{9-7.pdf}    
  \caption{}\label{fig:9-7}
\end{figure}

我们利用\cref{fig:9-7} 所示的电路来测定电流表的内电阻 $r_g$。
$R$ 可用 \qty{470}{k\ohm} 的电位器\footnote{电位器是一种可以连续调节电阻值的变阻器,常用作分压器。},$R'$ 是电阻箱,合上电键 $K_1$,调整电位器 $R$,使电流表指针偏转到满刻度(要注意,不应使通过电流表的电流超过它的满度电流值,以免把表烧坏),然后再合上电键 $K_2$,调整电阻箱 $R'$ 的阻值,使电流表指针偏转到正好是满刻度的一半。
当 $R$ 比$R'$ 大很多时,接入 $R$ 后,干路中电流变化不大,因此可以认为 $r_g=R'$。


测出 $r_g$ 后,再计算出电流表的满度电压 $U_g$。
然后算出把它改装为 \qty{2}{V} 的伏特表时,应该串联多大的电阻 $R_1$。
在电阻箱上取好阻值 $R_1$,把电流表跟电阻箱串联起来,就是一个量程是 \qty{2}{V} 的伏特表了。
\begin{figure}
  \begin{minipage}[b]{0.48\linewidth}\centering
    \includegraphics{9-8.pdf}
    \caption{}\label{fig:9-8}
  \end{minipage}
  \begin{minipage}[b]{0.48\linewidth}\centering
    \includegraphics{9-9.pdf}  
    \caption{}\label{fig:9-9}
  \end{minipage}
\end{figure}

最后把改装成的伏特表跟标准伏特表核对一遍。
实验电路如\cref{fig:9-8} 所示,$V$ 是标准伏特表,改变变阻器 $R_2$ 的触点,使 $V$ 的读数分别为 \qty{0.5}{V}、\qty{1}{V}、\qty{1.5}{V}、\qty{2}{V} 时,核对一下改装的伏特表的读数是否正确。
核对时要注意搞清楚改装后电流表上刻度的每一小格表示多大电压。
最后算出改装的伏特表满刻度时的百分误差。
例如改装的伏特表在满刻度 \qty{2}{V} 时,标准伏特表的读数为 \qty{2.1}{V},那么满刻度时的百分误差就是
\[\frac{|2.1-2|}{2.1}=4.8\%.\]

\section{用安培表和伏特表测定电池的电动势和内电阻}
这个实验是用安培表和伏特表测出电流和路端电压,再用闭合电路的欧姆定律来求出电动势和内电阻。
实验电路如\cref{fig:9-9} 所示。

我们知道,只要改变 $R$ 的阻值,测出两组 $I$、$U$ 的数据。代入方程组
\[\begin{split}
    \mathcal{E}&=U_1+I_1r,\\
    \mathcal{E}&=U_2+I_2r.\\
\end{split}\]
就可以求出电动势 $\mathcal{E}$ 和内电阻 $r$,这样做在原理上虽然很简单,但偶然误差却很大。

为了减小偶然误差,我们可以多测出几组 $I$、$U$ 的数据,求出几组 $\mathcal{E}$、$r$ 值,最后分别算出它们的平均值。
此外,物理实验中还经常用作图法,现在我们就来学习作图法。

利用变阻器 $R$ 测出几组 $I$、$U$ 值后,在坐标纸上以 $I$ 为横坐标,$U$ 为纵坐标,画出 $U$--$I$ 关系图象。
根据闭合电路的欧姆定律,$U=\mathcal{E}-Ir$,因此 $U$ 是 $I$ 的一次函数,它们的图象应该是一条直线。
你得出的是不是一条直线?
把这条直线延长,使它跟纵轴相交,这个交点有什么物理意义?
在图象中内电阻是怎样表示出来的?
你怎样利用自己作出的图象来得到电池的电动势和内电阻?

这里还要作一点说明,作图时要适当选取横坐标、纵坐标的比例和坐标的起点,使实验数据大致布满整个图纸,不要集中在一边或一角。
这个实验的 $U$ 值不宜过小,因此纵坐标 $U$ 的起点不要从零开始,而横坐标 $I$ 仍要以零为起点。(为什么?)

\section{练习使用万用电表}
万用电表(常简称为万用表)是一种多用仪表,一般可以用来测量电流、电压、电阻等,并且每一种测量项目有几个量程。
由于万用表具有用途多、量程广、使用方便等优点,因此得到了广泛的应用,这个实验我们来学习万用表的使用。

万用表的型号很多,但使用方法基本相同,下面以 J0411 型万用表为例来说明它的使用方法和注意事项。

\medskip\noindent
\begin{minipage}{0.47\linewidth}\parindent2em
J0411 型万用表的外形如\cref{fig:9-10} 所示。
它的上半部是表头,表盘上有电阻、电流、电压等各种量程的刻度。
有的刻度是均匀的,因此合用一个刻度。
下半部是选择开关,它的四周刻着各种测量项目和量程。
应当注意,电流和电压又分为直流(用符号“$-$”表示)和交流(用符号“$\oldsim$”表示),要区别开,不要弄错。
另外还有欧姆档的调零旋钮和测试笔插孔。

测量前,应先检查表针是否停在左端的“0”位置,否则,要用小螺丝刀轻轻地转动表盘下边中间的调整定位螺丝,使指针指零。
万用表有两根测试笔,将红表笔和黑表笔分别插入正($+$)、负($-$)测试笔插孔。
\end{minipage}\hfill
\begin{minipage}{0.48\linewidth}
  \begin{figurehere}
    \includegraphics{9-10.pdf}
    \caption{}\label{fig:9-10}
  \end{figurehere}
\end{minipage}

\medskip
测量时,应把选择开关旋到相应的项目和量程上。
读数时,要看跟选择开关的档位相应的刻度。

测量电流时,跟电流表一样,应把万用表串联在被测电路里;对于直流电,还必须使电流从红表笔流进万用表,从黑表笔流出来。

测量电压时,跟电压表一样,应把万用表和被测部分并联;对于直流电,必须用红表笔接电势较高的点,用黑表笔接电势较低的点。

测量电阻时,在选择好选择开关的档位后,要先把两根表笔相接触,调整欧姆档的调零旋钮,使指针指在电阻刻度的零位上(注意,电阻刻度的零位在表盘的右端)。
然后再把两表笔分别与待测电阻的两端相接,进行测量。
换用欧姆档的另一量程时,需要重新调整欧姆档的调零旋钮,才能进行测量。
应当注意,测量电阻时待测电阻要跟别的元件和电源断开。(为什么?)

测量时,注意手不要碰到表笔的金属触针,以保证安全和测量的准确;使用后,要把表笔从测试笔插孔拔出,井且不要把选择开关置于欧姆档,以防电池漏电;长期不使用时,应把电池取出。

在了解了你使用的万用表之后,就可以按照老师的要求,来进行电流、电压和电阻的测量了。

\section{用惠斯通电桥测电阻}
这个实验是用滑线式电桥来测电阻。

实验电路如\cref{fig:9-11} 所示,其中 $R_x$ 是待测电阻,$R_0$ 是作已知电阻用的电阻箱,$G$ 是灵敏电流表。
按图接好电路后,先把变阻器 $R$ 调到阻值较大的位置,然后进行实验。
\begin{figure}
  \includegraphics{9-11.pdf}  
  \caption{}\label{fig:9-11}
\end{figure}

根据误差理论,触头 $D$ 在 $AC$ 中点附近电桥平衡时实验误差较小(这个道理在这里就不讲了)。
我们先用万用表测出 $R_x$ 的大约值,在电阻箱上选取跟它接近的某一阻值 $R_0$。
合上电键 $K$,把滑动触头 $D$ 移到电阻线 $AC$ 中点附近某一位置,瞬时按下触头,一般会看到电流表的指针有偏转,稍稍移动触头,再把它瞬时按下,比较电流表指针两次偏转的情况。
根据指针偏转的方向是否相同和偏角是增大还是减小,你应该能判断出应向哪个方向移动触头才能使电桥平衡。
继续移动触头直到电桥平衡,电流表的指针不再偏转为止。
要注意,每次按下触头的时间要尽量短,用力不要过大,更不要在按下触头后又设法移动它。

电桥平衡后,打开电键 $K$,读出或量出 $AD$ 的长度 $l_1$ 和 $DC$ 的长度 $l_2$,根据 $R_0/R_x=l_1/l_2$ 求出 $R_x$,这就初步测出了$R_x$ 的值。

现在来进一步更精确地测定 $R_x$。
先在电阻箱上取跟初步测出的 $R_x$ 相近的阻值,重新使电桥平衡。
然后逐步减少变阻器 $R$ 的阻值,以增大 $AC$ 间的电压,但要注意通过电阻线 $AC$ 的电流不能超过它的允许值。
可以看到,每当 $R$ 的阻值减少后,按下触头$D$ 时电桥又可能不平衡了,每次都要重新调整触头 $D$ 的位置,才能使电桥恢复平衡。
同学们想想看,这是什么道理。
要注意这时每次都只能微调触头 $D$ 的位置,以免烧毁电流表。
当 $R$ 的阻值减小到一定程度时,使电桥平衡,然后读出或量出 $l_1$ 和 $l_2$,利用公式算出 $R_x$。
为什么现在求出的 $R_x$ 比初测的 $R_x$ 精确?

\section{测定铜的电化当量}
在这个实验里,我们根据法拉第电解第一定律 $m=kIt$,测出 $m$、$I$ 和 $t$ 的值,从而确定电化当量 $k$。

\medskip\noindent
\begin{minipage}{0.65\linewidth}\parindent2em
准备三块铜片,两块作为阳极,一块作为阴极,并用细砂纸把铜片擦干净,用天平仔细称量作为阴极的铜片的质量。
把铜片放入盛有硫酸铜溶液的电解槽内。
按照\cref{fig:9-12} 接好电路(注意电源和安培表的正负端不要接错),合上电键 $K$,调节变阻器 $R$ 使安培表的读数为 \qty{2}{A} 左右,并开始计时。
\qtyrange{25}{30}{min} 后,打开电键 $K$,停止电解。
注意要在整个电解过程中,调节变阻器使电流强度保持不变。
电解结束后,取出电极,用酒精灯把阴极板烘干,再用天平仔细称量出这时阴极板的质量。比较两次称量的阴极板的质量,就可以得到电解过程中在阴极板上析出的铜的质量 $m$。
把 $m$、$I$ 和 $t$ 带入法拉第电解第一定律公式,算出铜的电化当量。
\end{minipage}\hfill
\begin{minipage}{0.3\linewidth}\centering
  \begin{figurehere}
    \includegraphics{9-12.pdf}
    \caption{}\label{fig:9-12}
  \end{figurehere}
\end{minipage}

\medskip
你测定的铜的电化当量是多少?
跟课本上给出的数值相差多少?
考虑一下,实验误差的主要原因是什么?
应当怎样改进这个实验?

\section{练习使用示波器}
示波器是一种常用的电子仪器,它的核心部分是一只示波管,利用它能够直接观察电信号随时间而变化的情况,并且可以测量电压和电流。
我们现在初步学习一下示波器的使用方法,在后面的实验里还要多次用到它。

\medskip\noindent
\begin{minipage}{0.4\linewidth}\parindent2em
示波器的型号很多,使用方法基本相同,下面以 J2459 型示波器(\cref{fig:9-13})为例来说明。

我们先来了解示波器面板上各个旋纽和开关的名称和作用。
荧光屏右边最上端的旋钮是辉度调节“\faSun[regular]”,用来调节图象的亮度,顺时针旋转时亮度逐渐增大。
它下面的旋钮是聚焦调节“\faDotCircle[regular]”和辅助聚焦“\faCircle[regular]”,这两个旋钮配合着使用,能使电子射线会聚,在荧光屏上产生一个小的亮斑,得到清晰的图象。
再下面是电源开关和指示灯,用后盖板上的电源插座接通电源后,把开关板向“开”的位置,指示灯亮,经过一两分钟的预热,示波器就可以使用了。
\end{minipage}\hfill
\begin{minipage}{0.55\linewidth}\centering
  \begin{figurehere}
    \includegraphics{9-13.pdf}
    \caption{J2459 型示波器的面板}\label{fig:9-13}
  \end{figurehere}
\end{minipage}

\medskip
荧光屏下边第一行左右两端的旋钮是竖直位移“$\uparrow\downarrow$”和水平位移“$\leftrightarrows$”,分别用来调整图象在竖直方向和水平方向的位置。
它们中间的两个旋钮是“Y 增益”和“X 增益”,分别用来调整图象在竖直方向和水平方向的幅度,顺时针旋转时,幅度连续增大。

中间一行左边的大旋钮是“衰减”,它有 1, 10、100、1000 四档,最左边的“1”档不衰减,其余各档分别使输入电压衰减为原来的 1/10、1/100、1/1000,因此图象在竖直方向的幅度都减小为前一档的十分之一;最右边的正弦符号“\tikz \draw[x=.7ex,y=1ex] (0,0) sin (1,1) cos (2,0) sin (3,-1) cos (4,0)--(0,0);”档不是衰减,而是由示波器内部自行提供竖直方向的交流试验信号电压,可用来观察正弦波形或检查示波器是否正常工作。
右边的大旋钮是“扫描范围”,也有四档,可以改变加在水平方向的扫描电压的频率范围,左边第一档是 \qtyrange{10}{100}{Hz},向右旋转每升高一档,扫描频率都增大 10 倍,最右边的是“外 X”档,使用这一档时,机内没有加扫描电压,水平方向的电压可以从外部输入。
中间的小旋钮是“扫描微调”,用来调整水平方向的扫描频率,顺时针转动时频率连续增加。

底下一行中间的旋钮“Y 输入”、“X 输入”和“地”分别是竖直方向、水平方向和公共接地的输入接线柱。
左边的“DC、AC”是竖直方向输入信号的直流、交流选择开关。
置于“DC”位置时,所加的信号电压是直接输入的;置于“AC”位置时,所加的信号电压是通过一个电容器输入的,它可以让交流信号通过而隔断直流成分。
右边的“同步”也是一个选择开关,置于“$+$”位置时,扫描由被测信号正半周起同步,置于“$-$”位置时,扫描由负半周起同步。

现在,我们来练习使用示波器。
先把辉度调节旋钮反时针转到底,竖直位移和水平位移旋钮旋转到中间位置,衰减旋钮置于最高档,扫描范围旋钮置于“外 X”档,打开电源开关,指示灯亮。
经预热后,顺时针旋转辉度调节旋钮,屏上即出现一个亮斑。
亮斑的亮度要适中,注意不应使亮斑过亮,特别是当亮斑长时间停留在屏上不动时,应把亮度减弱,以免损伤荧光屏,减少示波管的使用寿命。
旋转聚焦调节和辅助聚焦旋钮,观察亮斑的变化,使亮斑最圆最小。
旋转竖直位移旋钮,观察亮斑的上下移动,旋转水平位移旋钮,观察亮斑的左右移动。

把 X 增益旋钮顺时针转到三分之一处,扫描微调旋钮反时针转到底,扫描范围旋钮置于最低档。
可以看到扫描的情形:亮斑从左向右移动,到右端后又很快回到左端,顺时针旋转扫描微调以增大扫描频率,可以看到亮斑迅速移动成为一条亮线。
调整 X 增益,可以看到亮线长度的改变。

现在给竖直方向加一个直流电压。事先把扫描范围旋钮置于“外 X”档,使亮斑位于屏的中心,把“DC、AC”开关置于“DC”位置。
照\cref{fig:9-14} 连接电路,直流电源用一、二节干电池即可。
逐步减小衰减档,观察亮斑的向上偏移,再调整 Y 增益使亮斑偏移一段适当的距离。
调整变阻器改变输入电压,可以看到亮斑的偏移随着改变,电压越高,偏移越大,调换电池的正负极,改变输入电压的方向,可以看到亮斑改为向下偏移。
\begin{figure}
  \includegraphics{9-14.pdf}
  \caption{}\label{fig:9-14}
\end{figure}

亮斑偏移的距离跟输入的电压成正比,因而利用示波器能够测量电压。
J2459 型示波器的竖直位移已经校准。
当衰减旋钮处在“1”的位置,Y 增益旋钮顺时针转到底时,如果输入电压为 \qty{50}{mV},则亮斑恰好偏移 1 格。
这样,我们就可以根据亮斑偏移的格数来算出输入的电压值,测量时要注意把 Y 增益旋钮顺时针转到底;衰减旋钮处在 10、100 或 1000 档时,算出的电压值应乘以相应的倍数。
现在来测一节干电池的电压,你测出的数值是多少?

利用示波器还可以测量电流。
把一个已知阻值的小电阻串联在待测电流的电路里(或利用原电路中的已知电阻),用示波器测量这个电阻两端的电压,利用欧姆定律就可以算出电路中的电流。
这种测量我们就不做了。

实验完了,关机前要注意把辉度调节旋钮反时针方向转到底,使亮度减到最小。